\chapter{Specyfikacja wewnętrzna}
\label{ch:05}
Kompilacja kodu podzielona jest na wiele etapów. Pozwala to na uproszenie skomplikowanych procesów i podzielenie ich odpowiedzialności. W skład procesów odbywających się w trakcie kompilacji wchodzą \cite{Aho2022Kompilatory}:
\begin{itemize}
\item analiza
\begin{itemize}
\item analiza leksykalna -- podział kodu programu na tokeny na podstawie zdefiniowanej gramatyki leksykalnej \cite{appelModernCompilerImplementation2002},
\item analiza syntaktyczna -- budowa drzewa składni abstracyjnej \cite{appelModernCompilerImplementation2002} (ang. \english{abstract syntax tree}) na podstawie gramatyki, poprzez konsumpcję tokenów \cite{appelModernCompilerImplementation2002},
\item analiza semantyczna -- kontrola drzewa składni syntaktycznej pod kątem m.in.: spójności typów danych, dozwolonych operacji, zdefiniowanych funkcji i zmiennych \cite{nystromCraftingInterpreters2021},
\item generowanie kodu pośredniego -- zamiana drzewa syntaktycznego na uniwersalny kod pośredni,
\end{itemize}
\item synteza
\begin{itemize}
\item optymalizacja -- proces modyfikacji kodu pośredniego zmniejszający zużycie zasobów i zwiększający szybkość wykonywania programu docelowego bez zmiany jego zachowania,
\item generowanie kodu docelowego -- zamiana kodu pośredniego do kodu instrukcji zrozumiałych dla platformy docelowej.
\end{itemize}
\end{itemize}

\section{Analiza leksykalna}
Do zaprojektowania analizatora leksykalnego wykorzystano narzędzie JFlex. Konsumuje ono specyfikację leksykalną zawartą w pliku definicji i generuje kod na jego podstawie kod analizatora leksykalnego w języku Java. Specyfikacja leksykalna znajduje się w pliku \lstinline|app/src/main/resources/UCricketLexer.y|. Wygenerowany kod analizatora leksykalnego umieszony zostaje w pliku \lstinline|app/src/main/java/pl/polsl/student/maciwal866/ucricket/UCricketLexer.java|.

Plik specyfikacji leksykalnej zaczyna się od konfiguracji narzędzia JFlex. Kod źródłowy, umieszczony w pierwszej linijce definicji, określa pakiet języka Java, w jakim znajduje się kod analizatora:
\begin{lstlisting}
package pl.polsl.student.maciwal866.ucricket;
\end{lstlisting}
Kod umieszczony w znacznikach, składających się z dwóch procentów \lstinline|%%|, zawiera główną konfigurację narzędzia JFlex (kod zródłowy na rys. \ref{fig:lst:lexer-konfiguracja}). Konfiguracja zawiera następujące elemnty:
\begin{itemize}
\item operatora \lstinline|%unicode| zmieniającego domyślne kodowanie znaków na Unicode,
\item operatora \lstinline|%standalone| pozwalającego na uruchomienie analizatora leksykalnego jako niezależny program, ułatwiając wyszukiwanie błędów,
\item operatora \lstinline|%public| ustawiającego dostęp do generowanej klasy na publiczny,
\item operatora \lstinline|%class UCricketLexer| zmieniającego nazwę klasy,
\item operatora \lstinline|%implements UCricketParser.Lexer| definiującego dziedziczenie po interfejse \lstinline|Lexer| zawartego w klasie analizatora syntaktycznego,
\item operatorów \lstinline|%line| i \lstinline|%column| włączających możliwość pobrania aktualnej lokalizacji karetki w analizowanym kodzie źródłowym,
\item bloku kodu \lstinline|%{ ... %}| zawierającego przesłaniane metody,
\item definicji wyrażeń regularnych wraz z ich identyfikatorami.
\end{itemize}
Do poprawnej pracy analizatora leksykalnego wymagane jest przesłonięcie dwóch metod. Metoda \lstinline|yyerror(String msg)| odpowiada za raportowanie błędów napotkanych w trakcie procesu analizy. Metoda \lstinline|getLVal()| jest metodą dziedziczoną z interfejsu \lstinline|Lexer| klasy analizatora syntaktycznego i jest ona wywoływana w trakcie jego pracy.
Reguły leksykalne opisane zostały pod blokiem konfiguracyjnym. Kod źródłowy na rys. \ref{fig:lst:lexer-reguły} przedstawia wybrane reguły definiujące tokeny jednoznakowe, tokeny wieloznakowe, słowa kluczowe i wyrażenia regularne. Na podstawie tych reguł generowane są tokeny przechwytywane przez analizator syntaktyczny.
Generowanie kodu analizatora leksykalnego obydwa się poprzez wywołanie komendy \lstinline|./gradlew app:buildLexer| z poziomu głównego katalogu projektu kompilatora.
\begin{lstlisting}
"+" {return '+';}
"==" {return UCricketParser.Lexer.EQUAL_EQUAL;}
\end{lstlisting}

\section{Analiza syntaktyczna}
GNU Bison  \cite{BisonGNUProject} jest zaawansowanym generatorem kodu analizatorów syntaktycznych. Wspiera on generowanie kodu analizatora do języków C, C++, D i Java poprzez dostarczenie pliku definicji syntaktycznych. Gramatyka dla narzędzia GNU Bison opisywana jest za pośrednictwem notacji Backusa-Naura (ang. \english{Backus-Naur Form}). Definicje syntaktyczne umieszczone są w pliku \lstinline|app/src/main/resources/UCricketParser.y|, natomiast wyjściowy kod analizatora generowany jest do pliku \lstinline|app/src/main/java/pl/polsl/student/maciwal866/ucricket/UCricketParser.java|.

Znacznik \lstinline|%token| odpowiada za definicję tokenów występujących w gramatyce. Tokeny te zwracane są w procesie analizy leksykalnej przez analizator leksykalny do analizatora syntaktycznego. Można je zaobserwować w specyfikacji leksykalnej, prezentowanej w kodzie źródłowym na rys. \ref{fig:lst:lexer-reguły}, w odwołaniach do interfejsu \lstinline|UCricketParser.Lexer|:
\begin{lstlisting}
%token IDENTIFIER INTEGER FLOAT IMPORT SCOPE IF WHILE FUNC VAR PTR RETURN TRUE FALSE EQUAL_EQUAL BANG_EQUAL LESS_EQUAL GREATER_EQUAL ASSIGN_ADDRESS
\end{lstlisting} 
Za pomocą znacznika \lstinline|%nterm| określa się typ klas wykorzystywancyh do budowy danych symboli nieterminalnych drzewa syntaktycznego. Kod źródłowy na rys. \ref{fig:lst:parser-nterms} przedstawia definicję symboli nieterminalnych występujących w języku uCricket.
Operator \lstinline|%start| pozwala na wskazanie, który symbol nieterminalny jest symbolem początkowym gramatyki.
Operatory \lstinline|%left| i \lstinline|%right| służą do zarządzania priorytetem i łącznością operacji w gramatyce. Kolejność definicji wskazuje poziom priorytetów od najniższego do najwyższego. Kod źródłowy na rys. \ref{fig:lst:parser-priorytety} przedstawia definicję priorytetów operacji, jakie wsytępują w języku uCricket.
W znacznikach \lstinline|%%| zawarta jest faktyczna definicja gramatyki. Definiowanie gramatyki odbywa się przy użyciu notacji Backusa-Naura poprzez podanie nazwy symbolu, dwukropka, definicji produkcji dla danego symbolu, klamr zawierających akcję semantyczną przypisaną do symbolu \ksremark{To jest translacja sterowana składnią.}, opcjonalnego znaku \lstinline/|/ odpowiedzialnego za definicję alternatywy i znaku średnika. Akcje definiowane w symbolach trafiają w zmodyfikowanej formie bezpośrednio do kodu wynikowego generowanego przez GNU Bison. Modyfikacji poddawane są wyrażenia rozpoczynające się od znaku dolara. Wyrażenie \lstinline|$$| odpowiada lewej stronie produkcji, natomiast wyrażenia z liczbą odpowiadają indeksowi symbolu prawej strony produkcji. Celem akcji jest tworzenie obiektów reprezentujących gałęzie drzewa syntaktycznego. Rysunek \ref{fig:lst:parser-gramatyka} przedstawia definicję gramatyki instrukcji (ang. \english{statements}) i wyrażeń (ang. \english{expressions}).
\begin{lstlisting}
binary:
        expression '+' expression { $$ = new BinaryExpression($1, BinaryExpression.Operator.ADD, $3); }
    |   expression '-' expression { $$ = new BinaryExpression($1, BinaryExpression.Operator.SUBTRACT, $3); }
;
\end{lstlisting}

W celu wygenerowania kodu analizatora dla języka, na podstawie istniejących definicji, należy użyć polecenia \lstinline|./gradlew app:buildParser|.

GNU Bison dla przedstawionej konfiguracji generuje analizator LALR(1) (ang. \english{lookahead left-to-right rightmost (derivation)}). Jest to jeden z typów analizatorów o mniejszej złożoności pamięciowej i większej elastyczności  \cite{baqaiComparisonParsingTechniques}.

\section{Analiza semantyczna}

Proces analizy semantycznej odbywa się na poziomie klas drzewa syntaktycznego implementowanych w języku Java. Do implementacji analizatora semantycznego wykorzystano wzorzec projektowy interpretera. Gałęzie drzewa syntaktycznego opierają się na dziedziczeniu odpowiednich interfejsów, które zwiększają ich funkcjonalność i definiują ich zachowanie. Interfejs \lstinline|ASTNode| określa, że dana klasa może być użyta jako gałąź struktury drzewa. Ponadto, klasy poddawane procesowi analizy semantycznej dziedziczą po interfejsie \lstinline|Resolvable|. Dostarcza on metodę \lstinline|resolve(Scope parent)| odpowiedzialną za proces analizy semantycznej. Zwraca ona także typ danych, zwracany przez wyrażenie, określany w trakcie analizy semantycznej. Rysunek \ref{fig:ast-uml} przedstawia graf \ksremark{diagram} dziedziczenia interfejsów i klas drzewa syntaktycznego \ksremark{Z tego diagramu nie wynika, że mamy drzewo. Żeby było drzewo, któryś z elementów powinien mieć wskaźnik (referencję) na ASTNode (lub inną klasę / interfejs). Czyli powinna pojawić się agregacja. Czy jest klasa dla scope?}. Proces analizy semantycznej odpowiada za:
\begin{itemize}
\item sprawdzanie istnienia zmiennych, funkcji i przestrzeni nazw w przypadku ich wywoływania,
\item weryfikację wolnych identyfikatorów w przypadku tworzenia zmiennych, funkcji i przestrzeni nazw,
\item statyczną kontrolę typów danych w trakcie kompilacji,
\item weryfikację kompatybilności typów danych w przypadku ich przemiennego użycia,
\item wiązanie pośrednie gałęzi pomiędzy strukturami definiującymi zmienne i funkcje w celu optymalizacji procesu kompilacji,
\item dynamiczne wykluczenie fragmentów drzewa syntaktycznego nieużywanych przez program w celu minimalizacji czasu kompilacji.
\end{itemize}
W przypadku niespełnienia któregoś z wymagań w procesie wytwarzania kodu, analizator semantyczny zgłosi odpowiedni wyjątek i przerwie proces kompilacji. Kod źródłowy na rys.  \ref{lst:resolver-function-call} przedstawia implementację metody \lstinline|resolve(Scope parent)| w klasie \mbox{\lstinline|FunctionCallExpression|.} W procesie prezentowanej metody wykonywane są:
\begin{itemize}
\item weryfikacja zgodności typów przekazywanych argumentów do wywołania funkcji,
\item sprawdzenie istnienia i dowiązanie obiektu funkcji do wyrażenia wywołania,
\item oznaczenie funkcji jako wywoływanej,
\item zwrócenie typu danych zwracanego przez funkcję.
\end{itemize}
\begin{figure}
\centering
\scalebox{0.7}{%
\begin{tikzpicture}
\begin{interface}{Resolvable}{0,3}
\operation{resolve(parent : Scoped) : Object}
\end{interface}

\begin{interface}{ASTNode}{0,0}
\inherit{Resolvable}
\end{interface}

\begin{interface}{Statement}{-3, -2}
\inherit{ASTNode}
\operation{solve(\\ builder : LLVMBuilderRef,\\ module : LLVMModuleRef,\\ context : LLVMContextRef)}
\end{interface}

\begin{interface}{Expression}{3, -2}
\inherit{ASTNode}
\operation{solve(\\ builder : LLVMBuilderRef,\\ module : LLVMModuleRef,\\ context : LLVMContextRef)\\ : LLVMValueRef}
\end{interface}

\begin{class}{IfStatement}{-9, -5}
\implement{Statement}
\end{class}

\begin{class}{AssignmentStatement}{-6, -7}
\implement{Statement}
\end{class}

\begin{class}{PointerStatement}{-3, -9}
\implement{Statement}
\end{class}

\begin{class}{FunctionCallExpression}{3, -9}
\implement{Expression}
\end{class}

\begin{class}{BinaryExpression}{6, -7}
\implement{Expression}
\end{class}

\begin{class}{ValueExpression}{9, -5}
\implement{Expression}
\end{class}

\end{tikzpicture}
}
\caption{Graf dziedziczenia interfejsów i klas drzewa syntaktycznego.}
\label{fig:ast-uml}
\end{figure}
\ksremark{Rys. \ref{fig:ast-uml}: graf → diagram.}

\begin{figure}
\begin{lstlisting}
@Override
public Object resolve(Scoped parent) {
    var argumentTypes = new ValueType[0];
    if (arguments != null && arguments.resolve(parent) instanceof ValueType[] resolvedArgumentTypes) {
        argumentTypes = resolvedArgumentTypes;
    }
    linkedFunction = parent.getFunction(functionName, argumentTypes);
    if (this.linkedFunction == null) {
        throw new FunctionNotFoundException(functionName, argumentTypes);
    }
    linkedFunction.setCalled(true);
    return linkedFunction.getType();
}
\end{lstlisting}
\caption{Metoda resolve(Scope parent) klasy FunctionCallExpression} \label{lst:resolver-function-call}
\end{figure}


\section{Generowanie kodu pośredniego}
Generowanie kodu pośredniego odbywa się z wykorzystaniem biblioteki LLVM wraz z~definicjami JavaCPP Presets dla LLVM. W procesie uczestniczy metoda \lstinline|solve(LLVMBuilderRef builder, LLVMModuleRef module, LLVMContextRef context)| dziedziczona z interfejsów, odpowiednio dla instrukcji i wyrażeń, \lstinline|Statement| i \lstinline|Expression|. Podział ten jest spowodowany obecnością referencji do struktury danych \lstinline|LLVMValueRef| zwracanej przez wyrażenia. Kod źródłowy na rys. \ref{fig:lst:ir-solve} prezentuje metodę \lstinline|solve()| definiującą reprezentację kodu pośredniego dla wywołania funkcji w języku uCricket. W tym procesie wyróżnia się:
\begin{itemize}
\item wyszukanie zdefiniowanej kodu pośredniego funkcji i oddelegowanie budowniczego w przypadku jej nieistnienia,
\item budowę struktur odpowiedzialnych za reprezentacje argumentów przekazywanych do funkcji,
\item definicję funkcji na poziomie budowniczego reprezentacji kodu pośredniego.
\end{itemize}
\begin{figure}
\begin{lstlisting}
    @Override
    public LLVMValueRef solve(LLVMBuilderRef builder, LLVMModuleRef module, LLVMContextRef context) {
        if (linkedFunction.getLlvmFunction() == null) {
            System.out.println("Delegating function build process to " + functionName + "().");
            var delegatedBuilder = LLVMCreateBuilderInContext(context);
            linkedFunction.solve(delegatedBuilder, module, context);
            LLVMDisposeBuilder(delegatedBuilder);
        }
        var argumentExpressions = new Expression[0];
        if (arguments != null) {
            argumentExpressions = arguments.collect();
        }
        var llvmArguments = new PointerPointer<LLVMTypeRef>(argumentExpressions.length);
        for (int i = 0; i < argumentExpressions.length; i++) {
            llvmArguments.put(argumentExpressions[i].solve(builder, module, context));
        }
        return LLVMBuildCall2(builder, linkedFunction.getLlvmFunctionType(), linkedFunction.getLlvmFunction(),
                llvmArguments, argumentExpressions.length,
                linkedFunction.getType().equals(ValueType.NONE) ? "" : functionName + "_ret");
    }
\end{lstlisting}
\caption{Metoda solve(LLVMBuilderRef builder, LLVMModuleRef module, LLVMContextRef context) klasy FunctionCallExpression}
\label{fig:lst:ir-solve}
\end{figure}
\section{Optymalizacja}
Optymalizacja odbywa się w trzech etapach kompilacji:
\begin{itemize}
\item analizy semantycznej -- wykluczenie fragmentów kody nie używanych w takcie pracy programu,
\item generowania kodu pośredniego -- optymalizacja wykonywanych operacji i redukcja wyrażeń możliwych do obliczenia bez potrzeby uruchamiania programu,
\item generowania kodu docelowego -- stosowanie przez kompilator kodu pośredniego optymalizacji specyficznych dla platformy.
\end{itemize}
Kompilator, w takcie generowania kodu pośredniego, stosuje domyślne optymalizacje włączone w narzędziu LLVM. W procesie generowania kodu docelowego, użytkownik ma możliwość zdefiniowania poziomu optymalizacji poprzez przekazanie do narzędzia \lstinline|avr-gcc| odpowiedniej flagi optymalizacji \lstinline|-Ox|, gdzie \lstinline|x| odpowiada poziomom optymalizacji zdefiniowanych w dokumentacji GNU GCC.

\section{Generowanie kodu docelowego}
Za generowaniu kodu docelowego dla platformy AVR odpowiada narzędzie \lstinline|avr-gcc|. Przekazywany jest do niego kod języka Assembly wygenerowany przez LLVM wraz z odpowiednimi flagami definiującymi platformę docelową. Kod źródłowy \ref{lst:gcc-compile} przedstawia komendę budującą kod binarny programu dla mikrokontrolera ATmega328P. Możliwe jest także podanie flag optymalizacyjnych kod wynikowy. W zależności od używanego programatora, wymagana jest konwersja pliku binarnego na format odpowiadający danym narzędziom.
\begin{lstlisting}[caption={Komenda kompilująca kod wynikowy z LLVM do kodu binarnego dla ATmega328P.}, label={lst:gcc-compile}]
avr-gcc -mmcu=atmega328p -o output.bin output.s
\end{lstlisting}

%Jeśli „Specyfikacja wewnętrzna”:
%\begin{itemize}
%\item przedstawienie idei
%\item architektura systemu
%\item opis struktur danych (i organizacji baz danych)
%\item komponenty, moduły, biblioteki, przegląd ważniejszych klas (jeśli występują)
%\item przegląd ważniejszych algorytmów (jeśli występują)
%\item szczegóły implementacji wybranych fragmentów, zastosowane wzorce projektowe
%\item diagramy UML
%\end{itemize}

% % % % % % % % % % % % % % % % % % % % % % % % % % % % % % % % % % % 
% Pakiet minted wymaga importu: \usepackage{minted}                 %
% i specjalnego kompilowania:                                       %
% pdflatex -shell-escape main                                       %
% % % % % % % % % % % % % % % % % % % % % % % % % % % % % % % % % % % 


%Krótka wstawka kodu w linii tekstu jest możliwa, np.  \lstinline|int a;| (biblioteka \texttt{listings})% lub  \mintinline{C++}|int a;| (biblioteka \texttt{minted})
%. 
%Dłuższe fragmenty lepiej jest umieszczać jako rysunek, np. kod na rys \ref{fig:pseudokod:listings}% i rys. \ref{fig:pseudokod:minted}
%, a naprawdę długie fragmenty – w załączniku.
%
%
%\begin{figure}
%\centering
%\begin{lstlisting}
%class test : public basic
%{
%    public:
%      test (int a);
%      friend std::ostream operator<<(std::ostream & s, 
%                                     const test & t);
%    protected:
%      int _a;  
%      
%};
%\end{lstlisting}
%\caption{Pseudokod w \texttt{listings}.}
%\label{fig:pseudokod:listings}
%\end{figure}
%
%%\begin{figure}
%%\centering
%%\begin{minted}[linenos,frame=lines]{c++}
%%class test : public basic
%%{
%%    public:
%%      test (int a);
%%      friend std::ostream operator<<(std::ostream & s, 
%%                                     const test & t);
%%    protected:
%%      int _a;  
%%      
%%};
%%\end{minted}
%%\caption{Pseudokod w \texttt{minted}.}
%%\label{fig:pseudokod:minted}
%%\end{figure}
%
%
