\chapter{Źródła}
\begin{figure}[H]
\begin{lstlisting}
%%
%unicode
%standalone
%public
%class UCricketLexer
%implements UCricketParser.Lexer
%line
%column

%{
    @Override
    public void yyerror(String msg){
        System.err.printf("%d:%d:%s\n%s\n", yyline + 1, yycolumn, yytext(), msg);
        System.exit(-1);
    }

    @Override
    public String getLVal(){
        return yytext();
    }
%}

COMMENT = \/\/.*
WHITESPACE = \ +|\n+|\r+|\t+
IDENTIFIER = [a-zA-Z][a-zA-Z0-9]*(:[a-zA-Z][a-zA-Z0-9]*)*
INTEGER = \d+
FLOAT = \d*\.\d+d

%%
\end{lstlisting}
\caption{Sekcja konfiguracyjna narzędzia JFlex}
\label{fig:lst:lexer-konfiguracja}
\end{figure}

\begin{figure}
\begin{lstlisting}
/* reguly tokenow jednoznakowych */
"+" {return '+';}
"-" {return '-';}
"*" {return '*';}

/* reguly tokenow wieloznakowych */
"==" {return UCricketParser.Lexer.EQUAL_EQUAL;}
"!=" {return UCricketParser.Lexer.BANG_EQUAL;}
"<=" {return UCricketParser.Lexer.LESS_EQUAL;} 

/* reguly slow kluczowych */
import {return UCricketParser.Lexer.IMPORT;}
scope {return UCricketParser.Lexer.SCOPE;}
func {return UCricketParser.Lexer.FUNC;}

/* reguly wyrazen regularnych */
{IDENTIFIER} {
    return UCricketParser.Lexer.IDENTIFIER;
}

{INTEGER} {
    return UCricketParser.Lexer.INTEGER;
}
\end{lstlisting}
\caption{Definicje reguł analizatora leksykalnego}
\label{fig:lst:lexer-reguły}
\end{figure}

\begin{figure}
\begin{lstlisting}
%nterm <ValueType> returnedType
%nterm <StatementChain> statements
%nterm <Statement> statement
%nterm <VariableStatement> variableStatement
%nterm <PointerStatement> pointerStatement
%nterm <Expression> expression binary unary primary functionCall condition
%nterm <ArgumentsExpression> arguments
%nterm <Scope> scope
%nterm <Function> function
%nterm <Scope.ScopeContent> scopeContent
%nterm <Function.ArgumentChain> argumentChain
\end{lstlisting}
\caption{Definicja symboli nieterminalnych analizatora syntaktycznego}
\label{fig:lst:parser-nterms}
\end{figure}

\begin{figure}
\begin{lstlisting}
%left EQUAL_EQUAL BANG_EQUAL LESS_EQUAL GREATER_EQUAL '<' '>' 
%left '&' '|' '^'
%left '+' '-'
%left '*' '/'
%right ARITHM_NEGATION LOGICAL_NEGATION
\end{lstlisting}
\caption{Definicja priorytetów i łączności operacji w analizatorze syntaktycznym}
\label{fig:lst:parser-priorytety}
\end{figure}

\begin{figure}
\begin{lstlisting}
%%
...
statements:
        statement statements { $$ = new StatementChain($1, $2); }
    |   { $$ = null; }
;

statement: 
        expression ';' {$$ = new ExpressionStatement($1); }
    |   IF '(' condition ')' '{' statements '}' { $$ = new IfStatement($3, $6); }
    |   WHILE '(' condition ')' '{' statements '}' { $$ = new WhileStatement($3, $6); }
    |   variableStatement { $$ = $1; }
    |   pointerStatement { $$ = $1; }
    |   IDENTIFIER '=' expression ';' { $$ = new AssignmentStatement(AssignmentType.VALUE, $<String>1, $3); }
    |   IDENTIFIER ASSIGN_ADDRESS expression ';' { $$ = new AssignmentStatement(AssignmentType.ADDRESS, $<String>1, $3); }
    |   RETURN expression ';' { $$ = new ReturnStatement($2); }
;

variableStatement:
        VAR returnedType IDENTIFIER ';' { $$ = new VariableStatement($2, $<String>3, null); }
    |   VAR returnedType IDENTIFIER '=' expression ';' { $$ = new VariableStatement($2, $<String>3, $5); }
;

pointerStatement:
        PTR returnedType IDENTIFIER ';' { $$ = new PointerStatement($2, $<String>3, AssignmentType.NONE, null); }
    |   PTR returnedType IDENTIFIER '=' expression ';' { $$ = new PointerStatement($2, $<String>3, AssignmentType.VALUE, $5); }
    |   PTR returnedType IDENTIFIER ASSIGN_ADDRESS expression ';' { $$ = new PointerStatement($2, $<String>3, AssignmentType.ADDRESS, $5); }
;
...
\end{lstlisting}
\caption{Przykładowy fragment gramatyki definiowanej w języku uCricket (część 1)}
\label{fig:lst:parser-gramatyka}
\end{figure} 

\begin{figure}
\begin{lstlisting}
expression:
        binary
    |   unary
    |   primary
    |   functionCall
;

binary:
        expression EQUAL_EQUAL expression { $$ = new BinaryExpression($1, BinaryExpression.Operator.EQUAL, $3); }
    |   expression BANG_EQUAL expression { $$ = new BinaryExpression($1, BinaryExpression.Operator.NOT_EQUAL, $3); }
    |   expression '<' expression { $$ = new BinaryExpression($1, BinaryExpression.Operator.LESS, $3); }
    |   expression '>' expression { $$ = new BinaryExpression($1, BinaryExpression.Operator.GREATER, $3); }
    |   expression LESS_EQUAL expression { $$ = new BinaryExpression($1, BinaryExpression.Operator.LESS_EQUAL, $3); }
    |   expression GREATER_EQUAL expression { $$ = new BinaryExpression($1, BinaryExpression.Operator.GREATER_EQUAL, $3); }
    |   expression '+' expression { $$ = new BinaryExpression($1, BinaryExpression.Operator.ADD, $3); }
    |   expression '-' expression { $$ = new BinaryExpression($1, BinaryExpression.Operator.SUBTRACT, $3); }
    |   expression '*' expression { $$ = new BinaryExpression($1, BinaryExpression.Operator.MULTIPLY, $3); }
    |   expression '/' expression { $$ = new BinaryExpression($1, BinaryExpression.Operator.DIVIDE, $3); }
    |   expression '&' expression { $$ = new BinaryExpression($1, BinaryExpression.Operator.AND, $3); }
    |   expression '|' expression { $$ = new BinaryExpression($1, BinaryExpression.Operator.OR, $3); }
    |   expression '^' expression { $$ = new BinaryExpression($1, BinaryExpression.Operator.XOR, $3); }
;
\end{lstlisting}
\caption{Przykładowy fragment gramatyki definiowanej w języku uCricket (część 2)}
\label{fig:lst:parser-gramatyka:2}
\end{figure} 

\begin{figure}
\begin{lstlisting}

unary:
        '(' expression ')' { $$ = new UnaryExpression($2, UnaryExpression.Type.PARENTHESIS); }
    |   '!' expression %prec LOGICAL_NEGATION { $$ = new UnaryExpression($2, UnaryExpression.Type.LOGICAL_NEGATION); }
    |   '-' expression %prec ARITHM_NEGATION { $$ = new UnaryExpression($2, UnaryExpression.Type.ARITHMETICAL_NEGATION); }
;
...
%%
\end{lstlisting}
\caption{Przykładowy fragment gramatyki definiowanej w języku uCricket (część 3)}
\label{fig:lst:parser-gramatyka:3}
\end{figure}

% % % % % % % % % % % % % % % % % % % % % % % % % % % % % % % % % % % 
% Pakiet minted wymaga odkomentowania w pliku config/settings.tex   %
% importu pakietu minted: \usepackage{minted}                       %
% i specjalnego kompilowania:                                       %
% pdflatex -shell-escape praca                                      %
% % % % % % % % % % % % % % % % % % % % % % % % % % % % % % % % % % % 

%\begin{minted}[linenos,breaklines,frame=lines]{c++}
%if (_nClusters < 1)
%   throw std::string ("unknown number of clusters");
%if (_nIterations < 1 and _epsilon < 0)
%   throw std::string ("You should set a maximal number of iteration or minimal difference -- epsilon.");
%if (_nIterations > 0 and _epsilon > 0)
%   throw std::string ("Both number of iterations and minimal epsilon set -- you should set either number of iterations or minimal epsilon.");
%\end{minted}
