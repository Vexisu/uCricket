\chapter{Weryfikacja i walidacja}
\label{ch:06}
Weryfikacja pracy kompilatora pod kątem poprawności generowanego kody wynikowego wymagała podzielenia procesu testowania na kilka etapów zależnych od implementowanej części kompilatora. W ich skład wchodzą:
\begin{itemize}
\item kontrola poprawności pracy analizy leksykalnej i syntaktycznej,
\item kontrola poprawności generowanego drzewa syntaktycznego,
\item weryfikacja pracy analizatora semantycznego,
\item weryfikacja poprawności generowanego kodu pośredniego,
\item weryfikacja poprawności generowanego kodu wynikowego dla docelowej platformy.
\end{itemize}
Ponadto, kod wynikowy był testowany na płytce rozwojowej Arduino Pro Mini wyposażonej w mikrokontroler ATmega328P.
\section{Kontrola analizatora leksykalnego i syntaktycznego}
Wykorzystanie języka Java do budowy kompilatora pozwoliło na łatwe kontrolowanie poprawności implementacji poprzez uprzednie zdefiniowanie interfejsów i klas. W efekcie, przy użyciu błędnych klas w definicjach leksykalnych lub syntaktycznych, kompilator języka Java zgłaszał stosowne błędy o niezgodności typów. Podejście to było pomocne szczególnie w trakcie implementacji gramatyki języka, ze względu na wymóg stosowania ściśle wybranych klas w konstruktorach elementów drzewa syntaktycznego.

Narzędzia JFlex i GNU Bison udostępniają dodatkowe techniki pomagające w rozwiązywaniu problemów. JFlex implementuje system kontroli poprawności składni zgłaszający odpowiednie wyjątki w przypadku błędów w definicji. Ponadto dzięki trybowi pracy niezależnej (ang. \english{standalone}), możliwe było zweryfikowanie poprawności generowanych tokenów poprzez podanie ich na wejściu programu.
GNU Bison prócz mechanizmowi zgłaszania wyjątków w trakcie budowania kompilatora, pozwala także na wygenerowanie reprezentacji reguł języka w formie tekstowej. Dzięki temu możliwa była identyfikacja konflików w implementowanej gramatyce i ich naprawa. Reprezentacja tekstowa była przydatna w szczególności podczas naprawy błędów związanych ze złym priorytetem operacji arytmetycznych.

\section{Poprawność drzewa syntaktycznego}
W celu sprawdzenia poprawności generowanego drzewa syntaktycznego przez analizator syntaktyczny wykorzystano wbudowany w środowisko programistyczne VSCodium debuger. Ze względu na obiektowe podejście do problemu budowy drzewa syntaktycznego, narzędzie to pozwoliło na łatwe przeglądanie aktualnie wygenerowanej struktury drzewa w formie listy. Proces kontroli drzewa syntaktycznego przedstawia rysunek \ref{fig:vscodium-debug-ast}.

\begin{figure}
\centering
	\includegraphics[width=1\textwidth]{graf/vscodium-debug-ast.png}
	\caption{Proces kontroli drzewa syntaktycznego za pomocą debugera wbudowanego w środowisko VSCodium.}
\label{fig:vscodium-debug-ast}
\end{figure}

\begin{itemize}
\item sposób testowania w ramach pracy (np. odniesienie do modelu V)
\item organizacja eksperymentów
\item przypadki testowe zakres testowania (pełny/niepełny)
\item wykryte i usunięte błędy
\item opcjonalnie wyniki badań eksperymentalnych
\end{itemize}

%\begin{table}
%\centering
%\caption{Nagłówek tabeli jest nad tabelą.}
%\label{id:tab:wyniki}
%\begin{tabular}{rrrrrrrr}
%\toprule
%	         &                                     \multicolumn{7}{c}{metoda}                                      \\
%	         \cmidrule{2-8}
%	         &         &         &        \multicolumn{3}{c}{alg. 3}        & \multicolumn{2}{c}{alg. 4, $\gamma = 2$} \\
%	         \cmidrule(r){4-6}\cmidrule(r){7-8}
%	$\zeta$ &     alg. 1 &   alg. 2 & $\alpha= 1.5$ & $\alpha= 2$ & $\alpha= 3$ &   $\beta = 0.1$  &   $\beta = -0.1$ \\
%\midrule
%	       0 &  8.3250 & 1.45305 &       7.5791 &    14.8517 &    20.0028 & 1.16396 &                       1.1365 \\
%	       5 &  0.6111 & 2.27126 &       6.9952 &    13.8560 &    18.6064 & 1.18659 &                       1.1630 \\
%	      10 & 11.6126 & 2.69218 &       6.2520 &    12.5202 &    16.8278 & 1.23180 &                       1.2045 \\
%	      15 &  0.5665 & 2.95046 &       5.7753 &    11.4588 &    15.4837 & 1.25131 &                       1.2614 \\
%	      20 & 15.8728 & 3.07225 &       5.3071 &    10.3935 &    13.8738 & 1.25307 &                       1.2217 \\
%	      25 &  0.9791 & 3.19034 &       5.4575 &     9.9533 &    13.0721 & 1.27104 &                       1.2640 \\
%	      30 &  2.0228 & 3.27474 &       5.7461 &     9.7164 &    12.2637 & 1.33404 &                       1.3209 \\
%	      35 & 13.4210 & 3.36086 &       6.6735 &    10.0442 &    12.0270 & 1.35385 &                       1.3059 \\
%	      40 & 13.2226 & 3.36420 &       7.7248 &    10.4495 &    12.0379 & 1.34919 &                       1.2768 \\
%	      45 & 12.8445 & 3.47436 &       8.5539 &    10.8552 &    12.2773 & 1.42303 &                       1.4362 \\
%	      50 & 12.9245 & 3.58228 &       9.2702 &    11.2183 &    12.3990 & 1.40922 &                       1.3724 \\
%\bottomrule
%\end{tabular}
%\end{table}  
%
