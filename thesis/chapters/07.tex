\chapter{Podsumowanie i wnioski}
\label{ch:07}
W ramach pracy dyplomowej powstał język programowania uCricket wraz z dedykowanym dla niego pełnym kompilatorem. Kompilator pozwala na kompilację kodu i uruchomienie go na platformie docelowej w postaci mikrokontrolera ATmega328P firmy Microchip.
Język programowania jest kompletny w sensie Turinga i pozwala programiście na:
\begin{itemize}
\item definiowanie funkcji,
\item definiowanie globalnych i lokalnych zmiennych i wskaźników,
\item wsparcie dla logicznych i liczbowych typów danych,
\item wykonywanie operacji arytmetyczno-logicznych,
\item organizację kodu poprzez mechanizm przestrzeni nazw,
\item kontrolę przepływu programu z wykorzystaniem instrukcji warunkowych,
\item definiowanie pętli,
\item pracę z wieloma plikami źródłowymi
\item komentowanie kodu.
\end{itemize}

Kompilator składa się z analizatora leksykalnego, analizatora syntaktycznego, analizatora semantycznego i generatora kodu pośredniego.
Analizator leksykalny dostarczany jest przez narzędzie JFlex. Przeprowadza on analizę leksykalną pliku wejściowego kodu źródłowego i generuje na ich podstawie tokeny. Analizator syntaktyczny opiera się na bibliotece GNU Bison, wykonując translację sterowaną składnią, wykorzystując metodę LALR(1). Analiza semantyczna została zaimplementowana w języku Java. Weryfikuje ona poprawność programu pod względem zgodności identyfikatorów i typów. Generator kodu pośredniego generuje kod programu możliwy do konwersji do pliku wykonywalnego dla mikrokontrolera ATmega328P.

Kompilator języka uCricket posiada optymalizacje, implementowane w kodzie kompilatora, dostarczane przez LLVM oraz GCC, zmniejszające złożoność pamięciową i obliczeniową programu wynikowego. Wykonuje on analizę kodu i redukuje go o nieużywane funkcje i zmienne. 
Dzięki wykorzystaniu narzędzi JFlex i GNU Bison możliwe jest proste rozszerzanie funkcjonalności języka. Ponadto, narzędzie LLVM wykonuje dodatkowe optymalizację kodu, specyficzne dla platformy docelowej, i pozwala na dodanie kolejnych platform wspieranych przez kompilator.

W dalszym toku pracy, projekt języka programowania może zostać rozwinięty o dodatkowe elementy gramatyczne, np.: wsparcie dla tablic, definicję kodu ewaluowanego w~ramach procesu kompilacji, dodatkowe operacje na wskaźnikach, wsparcie dla przekazywania referencji zmiennych między funkcjami.
Funkcjonalność kompilatora może zostać poszerzona o wsparcie dla innych modeli mikrokontrolerów i architektur. W konsekwencji dalszego rozwoju narzędzi infrastruktury kompilatorów LLVM, istnieje możliwość rezygnacji z wykorzystania narzędzia avr-gcc i integrację dodatkowych funkcjonalności LLVM do kompilatora języka uCricket. W momencie powstawania pracy, nie powstały jeszcze definicję interfejsu programistycznego dla zbioru dodatkowych narzędzi infrastruktury LLVM.

Proces projektowania oraz rozwoju języka programowania i kompilatora był procesem bardzo złożonym. Wymagał on zaznajomienia się z wieloma dziedzinami informatyki. Temat projektu poruszył zagadnienia języków formalnych, struktur danych, budowy kompilatorów, architektury komputerów, wraz z poddziedziną mikrokontrolerów i systemów wbudowanych, oraz inżynierii wstecznej. Nauczył on także sposobu organizacji pracy z~długoplanowym myśleniem o konsekwencjach podejmowanych decyzji projektowych.

