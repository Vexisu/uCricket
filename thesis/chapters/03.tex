\chapter{Wymagania i narzędzia}
\label{ch:03}

Zaprojektowanie języka programowania i kompilatora wymaga podjęcia odpowiednich decyzji projektowych i doboru adekwatnych narzędzi. Dotyczy to składni, sprawdzenia jej poprawności, transformacji do kodu maszynowego oraz ostatecznego tesowania w środowisku wirtualnym i docelowym.
\section{Wymagania projektowe}

Język programowania, będący tematem pracy, musi spełniać szereg założeń, pozwalających na wytwarzanie oprogramowania dla systemów wbudowanych z ograniczoną ilością zasobów pamięciowych i obliczeniowych. Ponadto język ma być zrozumiały dla początkujących programistów poprzez zastosowanie odpowiedniej semantyki i słów kluczowych.

\subsection{Wymagania funkcjonalne}
Spełnienie poniższych wymagań pozwoli na zaprojektowanie języka programowania, którego wykorzystanie będzie możliwe w budowie oprogramowania dla systemów wbudowanych. Wśród wymagań funkcjonalnych należy wymienić:
\begin{itemize}
\item kompilowalność do kodu maszynowego, specyficznego dla danej platformy,
\item dostęp do sprzętowych zasobów niskiego poziomu np.: adresy bezpośrednie pamięci, wskaźniki, zapis w rejestrach,
\item mechanizmy optymalizacji kodu zmniejszające złożoność pamięciową i obliczeniową programów,
\item ścisłe typowanie poprzez składnię języka i kontrola typów na etapie kompilacji,
\item mechanizmy abstrakcyjne ułatwiające organizacje kodu,
\item możliwość łączenia wielu plików kodu źródłowego z widoczną separacją pochodzenia zmiennych i funkcji,
\item kompletność w sensie Turinga.
\end{itemize}

\subsection{Wymagania niefunkcjonalne}
Głównymi wymaganiami niefunkcjonalnymi tego projektu są:
\begin{itemize}
\item przyjazność języka programowania dla początkujących programistów,
\item minimalizacja liczby błędów popełnianych przez programistę poprzez niskie skomplikowanie składni,
\item łatwość użycia kompilatora,
\item kompilator pracuje w środowiskach linuksowych.
\end{itemize}

\subsection{Istniejące rozwiązania}
Oficjalnie dostępne narzędzia dla mikrokontrolerów AVR, zapewniane przez firmę Microchip, pozwalają na programowanie w językach C, C++ i Assembly. Po mimo istnienia wspomnianych alternatyw, są one głównie stosowanymi narzędziami z powodu swojej dojrzałości.
Dzięki wykorzystaniu oficjalnej dystrybucji kompilatora GCC jako bazy, możliwe jest korzystanie z najnowszych standardów dostępnych w nim języków, wraz z identyczną składnią stosowaną w przypadku innych platform.

Zestaw bibliotek \lstinline|avr-gcc| dostarcza najpotrzebniejszych definicji elemntów mikrokontrolerów, wraz z wyróżnieniem ich model. Przykładowo, plik nagłówkowy \lstinline|avr/io.h| zarządza makrodefinicjami dla portów wejścia/wyjścia różnych modeli mikrokontrolerów. W przypadku mikrokontrolera ATmega328P, zestaw makrodefinicji znajduje się w pliku \lstinline|avr/iom328p.h|. Zawiera on zgodne z oficjalną dokumentacją makrodefinicje adresów portów we/wy, odpowiadających im bitów, rejestrów dostępu do rozszerzeń sprzętowych takich jak: przerwań, zarządzalnych zegarów, konwerterów analogowo-cyfrowych, sprzętowych interfejsów SPI/UART/TWI i wiele innych \cite{ATmega328P8bitAVR}.

Rysunek \ref{fig:lst:c-example} przedstawia przykładowy program włączający korespondujące diody podpięte do rejestru~C względem przełączników na rejestrze~D. Do implementacji tego programu wykorzystano plik nagłówkowy \lstinline|avr/io.h| dołączony do zbioru oficjalnych bibliotek dla mikrokontrolerów AVR.
\begin{figure}
\centering
\begin{lstlisting}
#include<avr/io.h> // plik makrodefinicji rejestrow

int main(){
	DDRC = 255; // rejestr C jako wyjscie
	DDRD = 0; // rejestr D jako wejscie
	while(1) {
		// przepisanie wartosci z rejestru D do C
		PORTC = PIND;
	}
}
\end{lstlisting}
\caption{Przykładowy program w języku C, wykorzystujący plik nagłówkowy \lstinline|avr/io.h|.}
\label{fig:lst:c-example}
\end{figure}

Możliwe jest także korzystanie w bibliotek dostarczanych przez Arduino. Wymaga to jednak odpowiedniej konfiguracji platformy docelowej i środowiska programistycznego.

\subsection{Język uCricket}
Język uCricket ma na celu zminimalizowanie popełnianych przez początkujących programistów błędów pojawiających się w trakcie wytwarzania oprogramowania poprzez zapewnienie stosownych mechanizmów prewencji. 
\subsubsection{Ściślejsza kontrola typów}
Jednym z takich mechanizmów jest dużo ściślejsza kontrola typów danych. Języki C/C++, w swoim  mechanizmie typowania, traktują wartości numeryczne i logiczne jako jeden typ, który można stosować zamiennie, co może prowadzić do niejasnego zachowania programu. 

\begin{lstlisting}
var<Integer> var1 = true; // Przypisanie wartosci boolowskiej do zmiennej liczbowej zlasza blad.
var<Boolean> var2 = 1; // Przypisanie wartosci liczbowej do zmiennej boolowskiej zglasza blad.
\end{lstlisting}
\subsubsection{Zwiększenie czytelności języka}
Czytelność języka, w porównaniu do oficjalnych rozwiązań, starano się także rozwiązać poprzez dodanie odpowiednich słów kluczowych, definiujących elementy złożone, w formie słów kluczowych, reprezentujących ich znaczenie (kod źródłowy na rys. \ref{fig:lst:ucricket-example}):
\begin{itemize}
\item \lstinline|scope| -- definicja odseparowanej przestrzeni nazw,
\item \lstinline|func| -- definicja funkcji,
\item \lstinline|var| -- definicja zmiennej,
\item \lstinline|ptr| -- definicja wskaźnika.
\end{itemize}

\subsubsection{Przestrzenie nazw}
Kod języka uCricket wymaga stosowania przestrzeni nazw, z myślą redukcji powtórzeń i konfliktów nazw, w szczególności przy dołączaniu kolejnych plików źródłowych. Zapewnia to większą elastyczność w nazywaniu elementów oraz ich separację poprzez dodanie warstwy abstrakcji. Użycie przestrzeni nazw nie wpływa na rozmiar kodu wynikowego.

Rysunek \ref{fig:lst:ucricket-example} przykładowy program wraz z fragmentem biblioteki \lstinline|AvrIO.uchirp| dołączonej do projektu. Program włącza diody podłączone do rejestru C, kiedy zostaną wciśnięte korespondujące przyciski na porcie D.
\begin{figure}
\centering
\begin{lstlisting}
// lib/avr/AvrIO.uchirp
scope AvrIO {
    ptr<Byte> PINC <- 38; // Znacznik z przypisanym adresem rejestru PINC
    ptr<Byte> DDRC <- 39; // Znacznik z przypisanym adresem rejestru DDRC
    ptr<Byte> PORTC <- 40; // Znacznik z przypisanym adresem rejestru PORTC
    ptr<Byte> PIND <- 41; // Znacznik z przypisanym adresem rejestru PIND
    ptr<Byte> DDRD <- 42; // Znacznik z przypisanym adresem rejestru DDRD
    ptr<Byte> PORTD <- 43; // Znacznik z przypisanym adresem rejestru PORTC
}

// main.uchirp
import lib:avr:AvrIO; // Dolaczenie pliku zrodlowego lib/avr/AvrIO.uchirp

scope Main {
	func main() {
		AvrIO:DDRC = 255; Zmiana kierunku pinow we/wy na wyjscie
		AvrIO:DDRD = 0; Zmiana kierunku pinow we/wy na wejscie
		while(true) 
        {
			AvrIO:PORTC = AvrIO:PIND; Przepisanie stanu pinow rejestru D do C
		}
	}
}
\end{lstlisting}
\caption{Przykładowy program w języku uCricket.}
\label{fig:lst:ucricket-example}
\end{figure}

\section{Narzędzia}

Do zaprojektowania i implementacji języka programowania uCricket wykorzystano następujące narzędzia:
\begin{itemize}
\item Java -- język programowania cechujący się obiektowością, czytelną składnią i zaawansowaną kontrolę typów, ułatwia on pracę z wieloma typami dziedziczącymi między sobą;
\item JFlex -- narzędzie generujące kod analizatora leksykalnego w języku Java, pozwala na korzystanie z wyrażeń regularnych, specyfikacja leksykalna definiowana jest w oddzielnym pliku \cite{JFlexFeatures};
\item GNU Bison -- zaawansowane narzędzie generujące kod analizatora składniowego na podstawie specyfikacji gramatycznej. Powszechnie wykorzystywany jako narzędzie pomocniczne w implmementacji kompilatorów. Posiada wsparcie dla języka Java;
\item LLVM -- zestaw bibliotek i narzędzi pozwalających na generowanie kodu pośredniego poprzez udostępnione interfejsy programistyczne. Wspiera wiele platform, m.in: x86, ARM, PowerPC, AVR. Dostarcza także mechanizmy optymalizacji kodu wynikowego \cite{LLVMCompilerInfrastructurea};
\item JavaCPP Presets LLVM -- biblioteka-most udostępniająca interfejsy dla biblioteki LLVM języka C poprzez wykorzystanie JNI (Java Native Interface);
\item avr-gcc -- kompilator języka C dla mikrokontrolerów AVR. Stosowany jest jako generator kodu maszynowego dla kodu generowanego przez LLVM;
\item avrdude -- program pozwalający na programowanie mikrokontrolerów AVR;
\item Cutter -- platforma do inżynierii wstecznej pomagająca w analizie wygenerowanego kodu maszynowego. Pracuje w oparciu o Rizin Framework;
\item Arduino Pro Mini -- płytka rozwojowa z mikrokontrolerem ATmega328P;
\item USBasp -- programator mikrokontrolerów AVR.
\end{itemize}

Zestaw tych narzędzi i bibliotek pozwala na projektowanie skompikowanych kompilatorów generujących niskopoziomowy, a także testowanie i weryfikacji poprawności programów napisanych w tworzonym języku programowania.

%\begin{itemize}
%\item wymagania funkcjonalne i niefunkcjonalne
%\item przypadki użycia (diagramy UML) -- dla prac, w których mają zastosowanie
%\item opis narzędzi, metod eksperymentalnych, metod modelowania itp.
%\item metodyka pracy nad projektowaniem i implementacją -- dla prac, w których ma to zastosowanie
%\end{itemize}
%