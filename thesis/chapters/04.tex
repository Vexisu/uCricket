\chapter{Specyfikacja zewnętrzna}
\label{ch:04}
Języki programowania składają się z części abstrakcyjnych nazywanych semantyką i gramatyką, oraz kompilatora. Semantyka i gramatyka opisuje, w jaki sposób należy zgodne z wymaganiami programy, natomiast kompilator zamienia program napisany w danym języku programowania na kod zrozumiały dla docelowej maszyny.
\section{Kompilator}
Dla każdego języka programowania powstaje dedykowane narzędzie odpowiedzialne za interpretację, sprawdzenie poprawności i wygenerowanie kodu wynikowego programu napisanego przez programistę.
\subsection{Wymagania}
Do uruchomienia kompilatora wymagane jest zapewnienie odpowiedniego środowiska uruchomieniowego składającego się z:
\begin{itemize}
\item systemu operacyjnego z rodziny GNU/Linux lub środowiska uruchomieniowego zapewniającego warunki pracy dla programów kompilowanych dla systemów GNU/Linux np. maszyna wirtualna, WSL, kontener Docker/Podman/Distrobox;
\item maszyny wirtualnej Java w wersji 17 lub wyższej;
\item zestawu narzędzi avr-gcc wraz z linkerem i assemblerem.
\end{itemize}

\subsection{Instalacja}
Kompilator dostarczany jest jako archiwum programu Java w formacie \lstinline|.jar|. Oprogramowanie LLVM zostało dołączone w archiwum jako biblioteka w formie binarnej dla platformy GNU/Linux.

\subsection{Użycie kompilatora}
Uruchomienie kompilatora odbywa się poprzez przekazanie maszynie wirtualnej programu kompilatora oraz dodaniu nazwy głównego pliku źródłowego bez rozszerzenia.
\begin{lstlisting}
java -jar uCricket.jar [glowny_plik_zrodlowy]
\end{lstlisting}
Kompilator zwraca w pliku \lstinline|output.s| program w języku assembly. Do zbudowania pliku binarnego, wykorzystywany jest \lstinline|avr-gcc|. Poniższe polecenie generuje plik \lstinline|output.bin| przeznaczony dla mikrokontrolera ATmega328P.
\begin{lstlisting}
avr-gcc -mmcu=atmega328p -o output.bin output.s
\end{lstlisting}
W celu wygenerowania pliku przeznaczonego do programowania przez narzędzie \lstinline|avrdude|, możliwe jest utworzenie pliku w formacie Intel HEX. Obywa się to poprzez wykorzystanie narzędzia \lstinline|avr-objcopy|. Poniższe polecenie generuje z pliku binarnego \lstinline|output.bin| plik \lstinline|output.hex|.
\begin{lstlisting}
avr-objcopy -j .text -j .data -O ihex output.bin output.hex
\end{lstlisting}
Programowanie mikrokontrolera kodem wynikowym możliwe jest dzięki wykonaniu poniższej komendy. W danym przykładzie użyty został programator USBasp wraz z mikrokontrolerem ATmega328P. W celu uzyskania uprawnień zapisu do pliku urządzenia programatora wymagana jest ich elewacja przy użyciu polecenia \lstinline|sudo|.
\begin{lstlisting}
sudo avrdude -p atmega328p -c usbasp -P usb -F -U flash:w:output.hex:i
\end{lstlisting}

\section{Język programowania}
Korzystanie z języków programowania wymaga znajomości ściśle zdefiniowanej semantyki i gramatyki oraz obsługiwanych typów danych i dozwolonych operatorów.
\subsection{Konwencja nazewnicza identyfikatorów}
Nazywanie elementów definiowanych przez użytkownika wymaga ścisłego używania nazw zgodnych z konwencją zapisu identyfikatorów. Identyfikatory przyjmują znaki alfanumeryczne, włącznie z dużymi i małymi literami, nie mogą zaczynać się cyfrą. Dana konwencja nazewnicza opisywana jest przez wyrażenie regularne \lstinline|[a-zA-Z][a-zA-Z0-9]*|.

\subsection{Pliki źródłowe}
Nazwy plików źródłowych muszą spełniać wymagania zapisu identyfikatorów. Nazwy te są wykorzystywane do identyfikacji plików źródłowych podczas dołączania źródeł między sobą i są niezależne od nazw przestrzeni (ang. \english{scopes}). Rozszerzeniem dla plików źródłowych jest \lstinline|.uchirp|.

\subsection{Typy danych}
Język programowania wspiera kilka głównych typów danych wykorzystywanych w innych językach programowania. Typy danych muszą być ściśle definiowane w trakcie trakcie ich używania. W innym przypadku, kompilator przerwie proces kompilacji w trakcie analizy statycznej. Tabela \ref{tbl:typy-danych} przedstawia typy danych wspierane przez język programowania.

\begin{table}
\centering
\caption{Typy danych obsługiwane przez język uCricket}
\begin{tabular}{@{}lllll@{}}
\toprule
typ danych & słowa & zakres & opis \\ \midrule
\lstinline|Boolean| & 1 & true/false & \begin{tabular}[c]{@{}l@{}}przechowuje wartość logiczną.\\ wykorzystywany w instrukcjach\\ warunkowych i pętlach.\end{tabular} \\
\lstinline|Byte| & 1 & [-127, 127] & typ liczby całkowitej ośmiobitowej. \\
\lstinline|Integer| & 2 & [-32767, 32767] & typ liczby całkowitej szesnastobitowej. \\
\lstinline|Float| & 4 & \begin{tabular}[c]{@{}l@{}}wg. standardu IEEE 754\\ dla czterech słów \cite{kahanIEEEStandard7541997}\end{tabular} & \begin{tabular}[c]{@{}l@{}}typ liczy zmienno-przecinkowej\\ trzydziestodwubitowej.\end{tabular} \\ \bottomrule
\end{tabular}
\label{tbl:typy-danych}
\end{table}
\subsection{Zmienne}
Zmienne mogą być definiowane jako globalne oraz lokalne w obszarze bloków funkcji, zmiennych i pętli . Typowanie zmiennych jest zawsze wymagane. Nie wymaga ono natomiast przypisania wartości. Definiowanie zmiennych odbywa się w następujący sposób:
\begin{lstlisting}
var<Boolean> boolValue = true;
var<Byte> byteValue = 12;
var<Integer> unassignedIntValue;
\end{lstlisting}
W trakcie przypisywania wartości do zmiennej możliwe jest użycie wyrażenia arytmetyczno-logicznego. Pozwala to także na zmianę wartości zmiennej z użyciem jej wcześniejszej wartości.
\begin{lstlisting}
var<Integer> i = 1;
i = i + 1; // Inkrementacja wartosci zmiennej i
\end{lstlisting}
\subsubsection{Zakresy zmiennych}
Pamięć alokowana przez zmienne jest zwalniana, gdy część wykonywana programu przesunie się poza klamry, np. po wyjściu z zakresu instrukcji warunkowej. Nie możliwe jest także odwoływanie się do zmiennych lokalnych poza ich zakresem.
\begin{lstlisting}
if (true){
	var<Integer> i = 0;
}
i = i + 1; // Blad nieistniejacej zmiennej.
\end{lstlisting}
\subsection{Wskaźniki}
Mechanizm wskaźników pozwala na bezpośredni dostęp do przestrzeni adresowej mikrokontrolera. Dzięki niemu możliwy jest także dostęp do rejestrów i portów wejścia wyjścia.
Definiowanie wskaźników odbywa się w sposób podobny do definicji zmiennych. Domyślnie, w odróżnieniu do zmiennych, definicja wskaźnika alokuje pamięć zmiennej i wpisuje jej adres do zmiennej wskaźnikowej. Odwołanie się do wskaźnika powoduje pośredni odczyt zmiennej poprzez jej adres, natomiast przypisanie wartości do zmiennej wskaźnikowej, wykonuje zapis do wskazywanego adresu pamięci. Poniższy przykład pokazuje definicję wskaźnika oraz inkrementację jego wartości.
\begin{lstlisting}
ptr<Integer> i = 0;
i = i + 1;
\end{lstlisting}
Z perspektywy programisty operacje na wskaźnikach wyglądają tak samo jak operacje na zmiennych.
\ksremark{Czy jest możliwe utworzenie drugiego wskaźnika do tego samego obszaru pamięci?} \textit{(Tak)}

Wskaźniki umożliwiają także na przypisanie oraz zmianę wartości bez alokacji pamięci właściwej. Wykorzystywany jest do tego operator \lstinline|<-| służący do przypisywania i zmiany adresów zmiennych wskaźnikowych. Poniższy przykład przedstawia przypisanie wartości do portów wejścia-wyjścia mikrokontrolera ATmega328P. Adresy dla portów B (\lstinline|PORTB|) i C (\lstinline|PORTC|) równe są ich lokalizacji w przestrzeni adresowej mikrokontrolera\cite{ATmega328P8bitAVR}.
\begin{lstlisting}
ptr<Byte> port <- 37; // Utworzenie wskaznika z adresem portu B.
port = 255; // Ustawienie wszystkich bitow na wartosc 1.
port <- 40; // Zmiana adresu na adres portu C.
port = 0; // Ustawienie wszystkich bitow na wartosc 0.
\end{lstlisting}
Operator \lstinline|<-| może być wykorzystywany tylko ze wskaźnikami. Jest to spowodowane różnicami działania wskaźników i zmiennych. Wskaźniki używają adresów pośrednich, zmienne natomiast przechowują swoje wartości w adresach bezpośrednich generowanych w trakcie procesu kompilacji.

\subsection{Operatory arytmetyczno-logiczne}
Język programowania uCricket implementuje wiele operatorów aryt\-me\-ty\-cz\-no-lo\-gi\-cz\-nych. Tabela \ref{tbl:operatory-arytmetyczno-logiczne} opisuje zbór operatorów obsługiwanych przez język wraz z przyjmowanymi typami danych. Tabela \ref{tbl:priorytety-operatorow} zawiera hierarchię priorytetów operatorów arytmetyczno-logicznych (ang. \english{precedence}). \ref{tbl:operatory-arytmetyczno-logiczne} i \ref{tbl:priorytety-operatorow}.

\begin{table}
\centering
\caption{Lista operatorów arytmetyczno-logicznych}
\begin{tabular}{lccp{2.8cm}p{2.8cm}}
\toprule
         &          &           & \multicolumn{2}{c}{typ} \\
\cmidrule{4-5}
nazwa operacji & operator & semantyka & przyjmowany & zwracany \\ \midrule
równy & \lstinline|==| & \lstinline|a == b| & liczbowy lub binarny & binarny \\
nierówny & \lstinline|!=| & \lstinline|a != b| & liczbowy lub binarny & binarny \\
mniejszy & \lstinline|<| & \lstinline|a < b| & liczbowy lub binarny & binarny \\
większy & \lstinline|>| & \lstinline|a > b| & liczbowy lub binarny & binarny \\
mniejszy lub równy & \lstinline|<=| & \lstinline|a <= b| & liczbowy lub binarny & binarny \\
większy lub równy & \lstinline|>=| & \lstinline|a >= b| & liczbowy lub binarny & binarny \\
dodawanie & \lstinline|+| & \lstinline|a + b| & liczbowe & taki jak lewy argument \\
odejmowanie & \lstinline|-| & \lstinline|a - b| & liczbowy & taki jak lewy argument \\
mnożenie & \lstinline|*| & \lstinline|a * b| & liczbowy & taki jak lewy argument \\
dzielenie & \lstinline|/| & \lstinline|a / b| & liczbowy & taki jak lewy argument \\
negacja arytmetyczna & \lstinline|-| & \lstinline|-a| & liczbowy & taki jak argument \\
koniunkcja & \lstinline|&| & \lstinline|a & b| & liczbowy lub binarny & taki jak lewy argument \\
alternatywa & \lstinline/|/ & \lstinline/a | b/ & liczbowy lub binarny & taki jak lewy argument \\
alternatywa rozłączna & \lstinline|^| & \lstinline|a ^ b| & liczbowy lub binarny & taki jak lewy argument \\
negacja logiczna & \lstinline|~| & \lstinline|~a| & liczbowy lub binarny & taki jak argument \\ \bottomrule
\end{tabular}
\label{tbl:operatory-arytmetyczno-logiczne}
\end{table}

\begin{table}
\centering
\caption{Priorytety operatorów arytmetyczno-logicznych (od najwyższego do najniższego)}
\begin{tabular}{lcl}
\toprule
priorytet & operatory & komentarz \\ \midrule
1 & \lstinline|- ~| & operatory jednoargumentowe \\
2 & \lstinline|* /| & \\
3 & \lstinline|+ -| & \\
4 & \lstinline/& | ^/ & \\
5 & \lstinline|== != < > <= >=| & \\ \bottomrule
\end{tabular}
\label{tbl:priorytety-operatorow}
\end{table}

\subsection{Instrukcje kontroli przepływu}
Aby zapewnić możliwość łatwego i czytelnego definiowania zachowań programu, języki programowania implementują różnego sposoby definiowania kontroli przepływu. Język uCricket implementuje następujące instrukcje pozwalające na taką kontrolę.

\subsubsection{Instrukcje warunkowe}
Instrukcja warunkowa pozwala na separacje bloku kodu i nadanie mu warunku, który spowoduje jego wykonanie. Definicja funkcji warunkowej odbywa się poprzez użycie słowa kluczowego \lstinline|if|, nawiasów okrągłych obejmujących warunek i nawiasów klamrowych zawierających blok kodu. Warunek musi zwracać wartość typu boolowskiego.
\begin{lstlisting}
if (true) {
	// Code
}
\end{lstlisting}

\subsubsection{Pętle}
Dzięki pętlom możliwe jest wielorazowe wykonywanie kodu, dopóki warunek jest spełniany. Petlę definiuje się poprzez użycie słowa kluczowego \lstinline|while|, nawiasów okrągłych obejmujących warunek i nawiasów klamrowych zawierających ciało pętli. Tak samo jak w przypadku instrukcji warunkowej  warunek przekazywany do pętli musi zwracać dane typu boolowskiego.
\begin{lstlisting}
while (true) {
	// Code
}
\end{lstlisting}
 
\subsection{Funkcje}
W języku uCricket istnieje kilka sposobów definicji funkcji:
\begin{itemize}
\item funkcja niezwracająca wartości, bez argumentów,
\item funkcja niezwracająca wartości, przyjmująca argumenty,
\item funkcja zwracająca wartość, bez argumentów,
\item funckja zwracająca wartość, przyjmująca argumenty. 
\end{itemize}

Podstawowa definicja funkcji składa się z słowa kluczowego \lstinline|func|, identyfikatora, otwierającego i zamykającego nawiasu okrągłego oraz nawiasów klamrowych będących ciałem funkcji.
\begin{lstlisting}
func foo() {
	// Code
}
\end{lstlisting}

Funkcje mogą przyjmować argumenty. Wartości przekazywane do funkcji są kopiowane do nowych zmiennych, definiowanych w nawiasach okrągłych. W przypadku przekazania do funkcji wskaźnika, przepisana zostanie wartość, na którą wskazuje wskaźnik.
\begin{lstlisting}
func foo(Byte byteBar, Integer intBar) {
	// Code
}
\end{lstlisting}

Poprzez dodanie po słowie kluczowym \lstinline|func| nawiasów ostrokątnych z typem danych, możliwe jest zdefiniowanie funkcji zwracającej wartość. Po dodaniu do funkcji typu danych wymagane jest zwrócenie wartości, o odpowiednim typie, na końcu funkcji. Odbywa się to poprzez użycie słowa kluczowego \lstinline|return|. Możliwe jest także zwrócenie wartości we wnętrzu funkcji, co spowoduje przerwanie jej wykonywania i przejście do punktu jej wywołania.
\begin{lstlisting}
func<Integer> add(Integer x, Integer y) {
	return x + y;
}
\end{lstlisting}

\subsection{Przestrzenie nazw}
Przestrzenie nazw pozwalają na separację funkcji i zmiennych globalnych w obrębie kodu. Ich stosowanie jest wymagane. Możliwe jest umieszczenie wielu przestrzeni nazw w jednym pliku źródłowym. Definiowanie przestrzeni nazw odbywa się w następujący sposób:
\begin{lstlisting}
scope NazwaPrzestrzeni {

}
\end{lstlisting}
W klamrach przestrzeni nazw możliwe jest definiowanie funkcji, zmiennych i wskaźników globalnych. 
\begin{lstlisting}
scope Globals {
	ptr<Byte> globalPointer <- 1;
	var<Byte> globalVariable = 1;
	func globalFunction() { 
		// Code	
	}
}
\end{lstlisting}

Aby odwołać się do zawartości innej przestrzeni, należy poprzedzić element właściwy danej przestrzeni nazwą przestrzeni zawierającej.
\begin{lstlisting}
scope Main {
	func main() {
		Aux:auxilary();	
	}
}

scope Aux {
	func auxilary(){
		// Code
	}
}
\end{lstlisting}
\ksremark{Można użyć przestrzeni nazw, która będzie zdefiniowana później?} \textit{Tak}

\subsection{Praca z wieloma plikami źródłowymi}
Pliki źródłowe mogą być łączone między sobą poprzez użycie słowa kluczowego \lstinline|import|:
\begin{lstlisting}
import katalog1:katalog2:NazwaPliku;
\end{lstlisting}
Katalogi w składni instrukcji \lstinline|import| oddzielane są dwukropkami i muszą spełniać wymagania zapisu identyfikatorów. Decyzję o zastąpieniu ukośnika występującą w składni drzewa katalogów podjęto w celu uspójnienia składni języka względem operatorów dostępu do przestrzeni nazw oraz zawartych w nich zmiennych i funkcji. Nazwę pliku źródłowego należy przekazywać do instrukcji bez rozszerzenia, jest ono dodawane w trakcie skanowania plików źródłowych. 

\subsection{Komentarze}
Dodawanie komentarzy do kodu odbywa się poprzez dodanie przed tekstem dwóch ukośników. Może on zostać dodany w dowolnym miejscu kodu. Komentarz kończy znak nowej linii.
\begin{lstlisting}
scope Main {
	// Komentarz do konca wiersza
	func main() {
		var<Integer> x = 3; // Komentarz do konca wiersza
	}
}
\end{lstlisting}

%\begin{itemize}
%\item  wymagania sprzętowe i programowe
%\item  sposób instalacji
%\item  sposób aktywacji
%\item  kategorie użytkowników
%\item  sposób obsługi
%\item  administracja systemem
%\item  kwestie bezpieczeństwa
%\item  przykład działania
%\item  scenariusze korzystania z systemu (ilustrowane zrzutami z ekranu lub generowanymi dokumentami)
%\end{itemize}

%%%%%%%%%%%%%%%%%%%%%
%% RYSUNEK Z PLIKU
%
%\begin{figure}
%\centering
%\includegraphics[width=0.5\textwidth]{./graf/politechnika_sl_logo_bw_pion_pl.pdf}
%\caption{Podpis rysunku zawsze pod rysunkiem.}
%\label{fig:etykieta-rysunku}
%\end{figure}
%Rys. \ref{fig:etykieta-rysunku} przestawia …
%%%%%%%%%%%%%%%%%%%%%
%
%%%%%%%%%%%%%%%%%%%%%
%% WIELE RYSUNKÓW 
%
%\begin{figure}
%\centering
%\begin{subfigure}{0.4\textwidth}
%    \includegraphics[width=\textwidth]{./graf/politechnika_sl_logo_bw_pion_pl.pdf}
%    \caption{Lewy górny rysunek.}
%    \label{fig:lewy-gorny}
%\end{subfigure}
%\hfill
%\begin{subfigure}{0.4\textwidth}
%    \includegraphics[width=\textwidth]{./graf/politechnika_sl_logo_bw_pion_pl.pdf}
%    \caption{Prawy górny rysunek.}
%    \label{fig:prawy-gorny}
%\end{subfigure}
%
%\begin{subfigure}{0.4\textwidth}
%    \includegraphics[width=\textwidth]{./graf/politechnika_sl_logo_bw_pion_pl.pdf}
%    \caption{Lewy dolny rysunek.}
%    \label{fig:lewy-dolny}
%\end{subfigure}
%\hfill
%\begin{subfigure}{0.4\textwidth}
%    \includegraphics[width=\textwidth]{./graf/politechnika_sl_logo_bw_pion_pl.pdf}
%    \caption{Prawy dolny rysunek.}
%    \label{fig:prawy-dolny}
%\end{subfigure}
%        
%\caption{Wspólny podpis kilku rysunków.}
%\label{fig:wiele-rysunkow}
%\end{figure}
%Rys. \ref{fig:wiele-rysunkow} przestawia wiele ważnych informacji, np. rys. \ref{fig:prawy-gorny} jest na prawo u góry.
%%%%%%%%%%%%%%%%%%%%%


 
%\begin{figure}
%\centering
%\begin{tikzpicture}
%\begin{axis}[
%    y tick label style={
%        /pgf/number format/.cd,
%            fixed,   % po zakomentowaniu os rzednych jest indeksowana wykladniczo
%            fixed zerofill, % 1.0 zamiast 1
%            precision=1,
%        /tikz/.cd
%    },
%    x tick label style={
%        /pgf/number format/.cd,
%            fixed,
%            fixed zerofill,
%            precision=2,
%        /tikz/.cd
%    }
%]
%\addplot [domain=0.0:0.1] {rnd};
%\end{axis} 
%\end{tikzpicture}
%\caption{Podpis rysunku po rysunkiem.}
%\label{fig:2}
%\end{figure}

