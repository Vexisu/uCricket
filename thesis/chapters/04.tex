\chapter{Specyfikacja zewnętrzna}
\label{ch:04}
\section{Kompilator}
\subsection{Wymagania}
Do uruchomienia kompilatora wymagane jest zapewnienie odpowiedniego środowiska uruchomieniowego składającego się z:
\begin{itemize}
\item systemu operacyjnego z rodziny GNU/Linux lub środowiska uruchomieniowego zapewniającego warunki pracy dla programów kompilowanych dla systemów GNU/Linux np. maszyna wirtualna, WSL, kontener Docker/Podman/Distrobox;
\item maszyna wirtualna Java w wersji 17 lub wyższej;
\item zestaw narzędzi avr-gcc wraz z linkerem i assemblerem.
\end{itemize}

\subsection{Instalacja}
Kompilator dostarczany jest jako archiwum programu Java w formacie .jar. Oprogramowanie LLVM zostało dołączone w archiwum jako biblioteka w formie binarnej dla platformy GNU/Linux.

\subsection{Użycie kompilatora}
Uruchomienie kompilatora odbywa się poprzez przekazanie maszynie wirtualnej programu kompilatora oraz dodaniu nazwy głównego pliku źródłowego bez rozszerzenia.
\begin{lstlisting}
java -jar uCricket.jar [glowny_plik_zrodlowy]
\end{lstlisting}

\section{Język programowania}

\begin{itemize}
\item  wymagania sprzętowe i programowe
\item  sposób instalacji
\item  sposób aktywacji
%\item  kategorie użytkowników
\item  sposób obsługi
\item  administracja systemem
\item  kwestie bezpieczeństwa
\item  przykład działania
%\item  scenariusze korzystania z systemu (ilustrowane zrzutami z ekranu lub generowanymi dokumentami)
\end{itemize}

%%%%%%%%%%%%%%%%%%%%%
%% RYSUNEK Z PLIKU
%
%\begin{figure}
%\centering
%\includegraphics[width=0.5\textwidth]{./graf/politechnika_sl_logo_bw_pion_pl.pdf}
%\caption{Podpis rysunku zawsze pod rysunkiem.}
%\label{fig:etykieta-rysunku}
%\end{figure}
%Rys. \ref{fig:etykieta-rysunku} przestawia …
%%%%%%%%%%%%%%%%%%%%%
%
%%%%%%%%%%%%%%%%%%%%%
%% WIELE RYSUNKÓW 
%
%\begin{figure}
%\centering
%\begin{subfigure}{0.4\textwidth}
%    \includegraphics[width=\textwidth]{./graf/politechnika_sl_logo_bw_pion_pl.pdf}
%    \caption{Lewy górny rysunek.}
%    \label{fig:lewy-gorny}
%\end{subfigure}
%\hfill
%\begin{subfigure}{0.4\textwidth}
%    \includegraphics[width=\textwidth]{./graf/politechnika_sl_logo_bw_pion_pl.pdf}
%    \caption{Prawy górny rysunek.}
%    \label{fig:prawy-gorny}
%\end{subfigure}
%
%\begin{subfigure}{0.4\textwidth}
%    \includegraphics[width=\textwidth]{./graf/politechnika_sl_logo_bw_pion_pl.pdf}
%    \caption{Lewy dolny rysunek.}
%    \label{fig:lewy-dolny}
%\end{subfigure}
%\hfill
%\begin{subfigure}{0.4\textwidth}
%    \includegraphics[width=\textwidth]{./graf/politechnika_sl_logo_bw_pion_pl.pdf}
%    \caption{Prawy dolny rysunek.}
%    \label{fig:prawy-dolny}
%\end{subfigure}
%        
%\caption{Wspólny podpis kilku rysunków.}
%\label{fig:wiele-rysunkow}
%\end{figure}
%Rys. \ref{fig:wiele-rysunkow} przestawia wiele ważnych informacji, np. rys. \ref{fig:prawy-gorny} jest na prawo u góry.
%%%%%%%%%%%%%%%%%%%%%


 
%\begin{figure}
%\centering
%\begin{tikzpicture}
%\begin{axis}[
%    y tick label style={
%        /pgf/number format/.cd,
%            fixed,   % po zakomentowaniu os rzednych jest indeksowana wykladniczo
%            fixed zerofill, % 1.0 zamiast 1
%            precision=1,
%        /tikz/.cd
%    },
%    x tick label style={
%        /pgf/number format/.cd,
%            fixed,
%            fixed zerofill,
%            precision=2,
%        /tikz/.cd
%    }
%]
%\addplot [domain=0.0:0.1] {rnd};
%\end{axis} 
%\end{tikzpicture}
%\caption{Podpis rysunku po rysunkiem.}
%\label{fig:2}
%\end{figure}

