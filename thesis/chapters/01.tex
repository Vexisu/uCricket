\chapter{Wstęp}
\label{ch:wstep}

Wytwarzanie oprogramowania dla systemów wbudowanych wymaga wiedzy specjalistycznej na wysokim poziomie. W przypadku tworzenia projektów amatorskich, przodującymi technologiami są płytki rozwojowe oparte o mikrokontrolery AVR serii ATmega. Ograniczenia tej architektury, tj. mała ilość pamięci operacyjnej i programowej, skłaniają programistów do korzystania z niskopoziomowych języków oraz bibliotek. Te czynniki są dużym utrudnieniem dla nowicjuszy, co w dużej mierze prowadzi do zniechęcenia do rozwoju w kierunku systemów wbudowanych, wydłuża czas realizacji projektów i zwiększa liczbę powstających błędów w wytwarzanym oprogramowaniu.

Celem tej pracy jest opracowanie języka programowania wraz z kompilatorem. Język ma zalety języków wysokopoziomowych, ułatwiającego pracę z mikrokontrolerami AVR. Docelowym układem, dla którego generowany oraz testowany będzie kod wynikowy, jest mikrokontroler ATmega328P, znany szeroko z występowania w płytkach rozwojowych Arduino Uno.

Język programowania uCricket będzie posiadał funkcjonalności znane z współcześnie wykorzystywanych języków programowania ułatwiających wytwarzanie oprogramowania dla systemów wbudowanych. Ponadto, ma on implementować mechanizmy zapobiegające występowaniu błędów w trakcie implementacji kodu, mające miejsce w przypadku użycia języków niskopoziomowych.

Kompilator będzie zbudowany z następujących części:
\begin{itemize}
\item analizatora leksykalnego,
\item analizatora syntaktycznego,
\item analizatora semantycznego,
\item generatora kodu pośredniego.
\end{itemize}
W procesie kompilacji kodu, kompilator będzie dokonywał stosownych optymalizacji, mających na celu redukcję złożoności pamięciowej i czasowej programów wynikowych.

Praca dyplomowa składa się z następujących rozdziałów. 
Rozdział \ref{ch:02} ,,Języki programowania dla mikrokontrolerów AVR'' przybliża historię platformy AVR, prezentuje teraźniejsze wykorzystanie mikrokontrolerów w projektach hobbystycznych oraz omawia istniejące rozwiązania. 
Rozdział \ref{ch:03} ,,Wymagania i narzędzia'' przedstawia wymagania i założenia kierujące rozwojem projektu. Omawia także narzędzia wykorzystane przy implementacji i przeprowadzonych testach kompilatora. 
Rozdział \ref{ch:04} ,,Specyfikacja zewnętrzna'' prezentuje sposób użycia praktyczne wykorzystanie implementowanych przez kompilator funkcjonalności.
Rozdział \ref{ch:05} ,,Specyfikacja wewnętrzna'' omawia zagadnienia teoretyczne występujące w procesie wytwarzania języków programowania oraz opisuje proces implementacji kompilatora języka uCricket. 
Rozdział \ref{ch:06} ,,Weryfikacja i walidacja'' przedstawia metodologię wykorzystaną podczas testowania poprawności pracy kompilatora. 
Rozdział~\ref{ch:07} ,,Podsumowanie i wnioski'' podsumowuje wyniki pracy związanej z tematem pracy dyplomowej. Omawia także możliwe do objęcia cele związane z przyszłym rozwojem języka programowania i kompilatora.

%\begin{itemize}
%\item wprowadzenie w problem/zagadnienie
%\item osadzenie problemu w dziedzinie
%\item cel pracy
%\item zakres pracy
%\item zwięzła charakterystyka rozdziałów
%\item jednoznaczne określenie wkładu autora, w przypadku prac wieloosobowych – tabela z autorstwem poszczególnych elementów pracy
%\end{itemize}