\chapter{Wstęp}
\label{ch:wstep}

\ksremark{test bibliografii: \cite{bib:artykul,bib:internet,bib:konferencja,bib:ksiazka}.}

Wytwarzanie oprogramowania dla systemów wbudowanych wymaga wiedzy specjalistycznej na wysokim poziomie. W przypadku tworzenia projektów amatorskich, przodującymi technologiami są płytki rozwojowe oparte o mikrokontrolery AVR serii ATMega. Ograniczenia tej architektury, tj. mała ilość pamięci operacyjnej i programowej, skłaniają programistów do korzystania z niskopoziomowych języków oraz bibliotek. Te czynniki są dużym utrudnieniem dla nowicjuszy, co w dużej mierze prowadzi do zniechęcenia do rozwoju w kierunku systemów wbudowanych, wydłuża czas realizacji projektów i zwiększa liczbę powstających błędów w wytwarzanym oprogramowaniu.

Celem tej pracy jest opracowanie języka programowania \ksremark{ wraz z kompilatorem. Język ma } zalety języków wysokopoziomowych, ułatwiającego pracę z mikrokontrolerami AVR. Docelowym układem, dla którego generowany oraz testowany będzie kod wynikowy, jest mikrokontroler ATMega328, znany szeroko z występowania w płytkach rozwojowych Arduino Uno. Ze względu na rozmiar rodziny mikrokontrolerów AVR, kompilator będzie umożliwiał wprowadzenie parametrów konfiguracyjnych, pozwalając tym samym na wsparcie dla większości członków tej rodziny mikrokontrolerów. 

\ksremark{Jakie cechy będzie miał język? Jak będzie zbudowany kompilator (3 analizatory), tył kompilator (optymalizacje)}

Opis rozdziałów będzie gotowy po bliższym przygotowaniu listy rozdziałów :)

\ksremark{Praca składa się z następujących rozdziałów. Języki programowania dla mk  przybliza rozzdiał \ref{ch:02} ... }

%\begin{itemize}
%\item wprowadzenie w problem/zagadnienie
%\item osadzenie problemu w dziedzinie
%\item cel pracy
%\item zakres pracy
%\item zwięzła charakterystyka rozdziałów
%\item jednoznaczne określenie wkładu autora, w przypadku prac wieloosobowych – tabela z autorstwem poszczególnych elementów pracy
%\end{itemize}