
%%%%%%%%%%%%%%%%%%%%%%%%%%%%%%%%%%%%%%%%%%%%%%%
%                                             %
% MOJE PAKIETY, USTAWIENIA ITD                %
%                                             %
%%%%%%%%%%%%%%%%%%%%%%%%%%%%%%%%%%%%%%%%%%%%%%%

% Tutaj proszę umieszczać swoje pakiety, makra, ustawienia itd.


 
%%%%%%%%%%%%%%%%%%%%%%%%%%%%%%%%%%%%%%%%%%%%%%%%%%%%%%%%%%%%%%%%%%%%%
% listingi i fragmentu kodu źródłowego 
% pakiet: listings lub minted
% % % % % % % % % % % % % % % % % % % % % % % % % % % % % % % % % % % 

% biblioteka listings
\usepackage{listings}
\lstset{%
language=C++,% C, Matlab, Python, SQL, TeX, XML, bash, ... – vide https://www.ctan.org/pkg/listings
commentstyle=\textit,%
identifierstyle=\textsf,%
keywordstyle=\sffamily\bfseries, %\texttt, %
%captionpos=b,%
tabsize=3,%
frame=lines,%
numbers=left,%
numberstyle=\tiny,%
numbersep=5pt,%
breaklines=true,%
escapeinside={@*}{*@},%
morekeywords={string,exception,std,vector,var,Integer,ptr,scope,func, Byte,import,Boolean},% słowa kluczowe rozpoznawane przez pakiet listings
}

\usepackage{float}
% % % % % % % % % % % % % % % % % % % % % % % % % % % % % % % % % % % 
% pakiet minted
%\usepackage{minted}

% pakiet wymaga specjalnego kompilowania:
% pdflatex -shell-escape main.tex
% xelatex  -shell-escape main.tex

%\usepackage[chapter]{minted} % [section]
%%\usemintedstyle{bw}   % czarno-białe kody 
%
%\setminted % https://ctan.org/pkg/minted
%{
%%fontsize=\normalsize,%\footnotesize,
%%captionpos=b,%
%tabsize=3,%
%frame=lines,%
%framesep=2mm,
%numbers=left,%
%numbersep=5pt,%
%breaklines=true,%
%escapeinside=@@,%
%}

%%%%%%%%%%%%%%%%%%%%%%%%%%%%%%%%%%%%%%%%%%%%%%%%%%%%%%%%%%%%%%%%%%%%%



%%%%%%%%%%%%%%%%%%%%%%%%%%%%%%%%%%%%%%%%%%%%%%%
%                                             %
% KONIEC MOICH USTAWIEŃ                       %
%                                             %
%%%%%%%%%%%%%%%%%%%%%%%%%%%%%%%%%%%%%%%%%%%%%%%

