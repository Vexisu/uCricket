\documentclass[12pt]{article}
\usepackage[british,polish]{babel} % Dodanie pakietu dla języka polskiego powoduje, że słowa będą dzielone wg reguł j. polskiego.
\usepackage[T1]{fontenc}
\usepackage[utf8]{inputenc}

%%%% TODO LIST GENERATOR %%%%%%%%%
\usepackage{color}
\definecolor{brickred}      {cmyk}{0   , 0.89, 0.94, 0.28}

\makeatletter \newcommand \kslistofremarks{\section*{Uwagi} \@starttoc{rks}}
  \newcommand\l@uwagas[2]
    {\par\noindent \textbf{#2:} %\parbox{10cm}
{#1}\par} \makeatother


\newcommand{\ksremark}[1]{%
{%\marginpar{\textdbend}
{\color{brickred}{[#1]}}}%
\addcontentsline{rks}{uwagas}{\protect{#1}}%
}
%%%%%%%%%%%%%% END OF TODO LIST GENERATOR %%%%%%%%%%%

\begin{document}
Celem dyplomu jest zaprojektowanie języka programowania oraz napisanie kompilatora dla mikrokontrolera w architekturze AVR. Język ten ma spełniać cechę kompletności Turinga, posiadać znane dla współczesnych języków programowania funkcjonalności, systemy zapobiegające powstawaniu błędów w programie (np. \foreignlanguage{british}{\emph{garbage collector}}) i~podstawowe algorytmy optymalizacji zmniejszające kod wynikowy oraz czas wykonania programów.
\end{document}
